
\documentstyle[11pt]{article}
\pagestyle{empty}

\setlength{\textwidth}{8in}
\setlength{\oddsidemargin}{0 in}
\setlength{\evensidemargin}{0 in}
\setlength{\topmargin}{20 pt}
\voffset=-1in
\hoffset=-0.3in


\begin{document}

\begin{center}
{\large \sf
Sample solution for NTU AD 711R 
Summer 2005
}
\end{center}



\begin{enumerate}
\item (60)
 Remember that your equation should be based on the first $n$ months not just for first 12 months.
 Also be careful when you draw the equation.
 \begin{enumerate}
  \item (40) Using the aggregate method, \\
   First, determine an upper bound on the sum of the costs for the first $n$ months. \\
   The sum of the actual monthly costs for the first $n$ months is
   \begin{eqnarray*}
& = & 80 \times \lfloor n/12 \rfloor + 40 \times (\lfloor n/3 \rfloor - \lfloor n/12 \rfloor) +
20 \times (n - \lfloor n/3 \rfloor) \\
& = & 40 \times \lfloor n/12 \rfloor + 20 \times \lfloor n/3 \rfloor + 20 \times n \\
& \le & 40 \times n/12 + 20 \times n/3 + 20 \times n \\
& = & 20 \cdot (1/6 + 1/3 + 1) \\
& = & 20 \cdot (3/2) \\
& = & 30 \cdot n \\
   \end{eqnarray*}
 
  \item (20) Using the amortized cost calculated in part(a), fill in the following
    table.\vspace{2mm}
\begin{table} [t,h]
\centerline{
\begin{tabular}{|c||c|c|c|c|c|c|c|c|c|c|c|c|}
\hline
Month & Jan. & Feb. & Mar. & Apr. & May & Jun. & Jul. & Aug. & Sep. & Oct. & Nov. & Dec.\\
\hline
\hline
Actual cost & 20 & 20 & 40 & 20 & 20 & 40 & 20 & 20 & 40 & 20 & 20 & 80 \\
\hline
Amortized cost & 30 & 30 & 30 & 30 & 30 & 30 & 30 & 30 & 30 & 30 & 30 & 30 \\
\hline
Potential() & 10 & 20 & 10 & 20 & 30 & 20 & 30 & 40 & 30 & 40 & 50 & 0 \\
\hline
\end{tabular}}
\vspace{2mm}
\caption{\label{table 1} {Maintenance Contract}}
\end{table}
\end{enumerate}

\item (60)
  S=180 records \\
  n=1000 records \\
  m runs \\
  $t_s$ = 6 ms \\
  $t_l$ = 4 ms \\
  $t_t$ = 0.1 ms / record

  \begin{enumerate}
   \item  $b = \lfloor S / (2k+2) \rfloor = \lfloor 180/ (2*8+2) \rfloor
          = 180/18 =10 $
   \item  time to read a buffer
           = ( 6 + 4 + (10)*0.1 ) = (10 + (10)*0.1) ms
           = 11 ms
   \item number of buffers per pass
           = $\lceil n/b \rceil = \lceil 1000 / 10 \rceil = 100 $
   \item input time per pass
           $ = (b) * (c) = 11*100 = 1100ms $
   \item number of passes
           $ = \lceil \log_{k} m  \rceil = \log_{8} m $
   \item total input time
           $ = (d) * (e) = 1100* \log_{8} m $
  \end{enumerate} 


\item (40)
 \begin{enumerate}
  \item (15)
   \begin{verbatim} 
                (3,40)                      
             /          \                    
        (4,30)          (5,37)             
        /    \          /    \                                 
     (9,25)  (11,29) (6,24)  (31)              
   \end{verbatim}

  \item (25) $RemoveMin()$ \\
    Remove $2$ from the root. \\ 
    Remove $37$ from the last node and insert it into the root. \\
   
   \begin{verbatim}
                (37,59)                      
             /          \                    
        (3,40)          (7,51)             
        /    \          /                                    
    (11,30)  (15,29) (8,42)          

 

                (3,59)                      
    ===>     /          \                    
       (37,40)          (7,51)             
        /    \          /                                    
    (11,30)  (15,29) (8,42)        



                (3,59)                      
    ===>     /          \                    
       (11,40)          (7,51)             
        /    \          /                                    
    (37,30)  (15,29) (8,42)     



    swap        (3,59)                      
    ===>     /          \                    
       (11,40)          (7,51)             
        /    \          /                                    
    (30,37)  (15,29) (8,42)     
   \end{verbatim}
 \end{enumerate}

\item (30) 
  Remember the definition of the height-biased min leftist tree. \\
    - has min-heap property\\
    - swap only if needed\\
  Meld right subtree with smaller root and all of the other tree.
   \begin{verbatim}

     meld[ 7 ,      4 ]          meld[10,  7 ]            7
          /       /   \   ===>        /   /      ===>   /   \
         8       15   10             21  8             8     10
                /     /                                     / 
               25    21                                    21 

                4             swap         4
     ===>    /     \          ===>      /     \
            15     7                   7      15
           /      /  \                /  \    / 
          25     8   10              8   10  25
                     /                   /
                    21                  21
 

                  2 
               /     \ 
     ===>    5        4 
           /   \   /     \
         12    9  7      15   
                  /  \   /    
                 8   10 25    
                    /           
                   21       
   \end{verbatim}


\item (60)
 \begin{enumerate}
 \item (10) 
  \begin{verbatim}
           min
            |
            v
            3  --- 9 --- 15 --- 8
          /  |
         5   7
             |
             10
  \end{verbatim}
 \item (20) Pairwise combine after deletion 
  \begin{verbatim}
          min
        |
        v
        5  ---- 15
      /  |      
     9   7    
         |
         10
  \end{verbatim}
  \item (30)
    create and initial tree table = $O(MaxDegree)$. \\
    Examine $t$ min trees and pairwise conbime = $O(t)$ \\
    Collect remaining trees from tree table, reset table entire to null  = $O(MaxDegree)$ \\
    Thus, overall complexity of Remove min = $O(MaxDegree + t)$

 \end{enumerate}

\end{enumerate}
\end{document}



